I was thinking I'd do a section for each metric/relationship I'm able to analyse

\section{Localisation Error Over Time}
start with description of the data being displayed on the charts and how the charts were generated (how data was processed and the technologies used)

\bigskip

Here I was thinking a graph or set of line charts displaying mean error rate for all participants over time for each source position, but twelve lines on a single chart might be too much? Unless it's one very large chart. So I was expecting to split it into three, one for each elevation - sources below the listener, sources at the same level, and sources above. There will also be only about 5 points along the X axis so they might all fit in a single line. Is there a well-defined style? Or should I just try both and go with whatever I think looks nicer?

\bigskip 

Then describe what the conclusions can be drawn from the data being displayed

\section{Relationships Between PCs and Localisation Errors}
Again, starting with a description of how the data was was collated from the tests, and how it was arranged on the charts. I just can't yet work out the best way of displaying the data for this. It would be nice to represent it on a sphere or section of a sphere, something like that? 

\bigskip

same structure, charts in between 

\bigskip 

Conclusions we can draw, etc

\section{•}
Then this structure could be repeated as needed, for as much analysis as I get to do??