Based on the research outlined in the previous chapter, it was decided that an initial implementation of the proposed process would involve modelling the HRTF using principal components analysis to reduce limit the number of variables in play to those that contain the greatest variance. Modifications to this reduced dataset would then be made based on simulated annealing search - or some variation of it. The efficacy of this implementation of this proposed approach will then be evaluated through listening tests conducted within a simple virtual reality environment. The metric for success is the same as the heuristic being used in the individualisation process - does the participant's ability to localise sound sources get better over time, and if so by how much?

\subsubsection{Implementation Questions}
For this implementation I elected to follow the model for PCA, along with PCW weight adjustment, outlined by Holzl\citep{Holzl2014a}. The core implementation would largely follow his formulation, as mapped out in the paper, with modifications where necessary. Having a method like this that allows for adaptation of the entire HRTF at once helps to simplify the calculations that need to take place during the individualisation process, a convenience when a lot of the processing is taking place in real time. The input matrix structure that was decided upon during Holzl's investigation, [(Directions x Subjects)(Frequencies x 2)], was modified slightly to work for a single user to become: [Directions x (Frequencies x 2)]. In practical terms, an entry from the CIPIC HRIR database would be transformed into the frequency domain, resulting in an HRTF dataset in the form [Left/Right (2) x Azimuths (25) x Elevations (50) x Frequencies (101)]. This is then restructured into the above form, resulting in a structure [(Azimuths, Elevations (1250)) x (Frequencies, Left/Right (202))] in size. This structure is intuitive in terms of how the original values map to the new one, a quality that carries over even after the structure has been transformed with PCA. Performing PCA on this matrix singles out the frequency bins that contain the most variance over all source directions. The resulting [PCW x PC] matrix maps to directly [Directions x PCs] where the PCs are the frequencies of greatest variance and the directions are the PCWs. The benefit of this resulting structure is that it becomes very easy to modify the PCWs that relate to the position of a sound source. So for my implementation this means that it is simple to map the degree to which a user is able to locate a sound source to a relevant PCW within each PC. Because of this ability to match sound sources with principal component weights, coupled with the fact that PCW modifications were to be automated rather than performed manually, I chose not to model the resulting PCWs using spherical harmonics, further simplifying the individualisation process. 

allows 10 PCs for 90 percent reconstruction


it is worth noting that my proposed search method is essentially running many individual searches - either in sequence or in parallel - in order to find the optimal value for *every source position*