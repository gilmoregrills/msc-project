Based on the research outlined in the previous chapter, it was decided that an initial implementation of the proposed process would involve modelling the HRTF using principal components analysis to reduce limit the number of variables in play to those that contain the greatest variance. Modifications to this reduced dataset would then be made based on simulated annealing search - or some variation of it. The efficacy of this implementation of this proposed approach will then be evaluated through listening tests conducted within a simple virtual reality environment. The metric for success is the same as the heuristic being used in the individualisation process - does the participant's ability to localise sound sources get better over time, and if so by how much?

\subsubsection{Process Notes}
For this implementation I elected to follow the model for PCA, along with PCW weight adjustment, outlined by Holzl\citep{Holzl2014a}. The core implementation would largely follow his formulation, as mapped out in the paper, with modifications where necessary. Having a method like this that allows for adaptation of the entire HRTF at once helps to simplify the calculations that need to take place during the individualisation process, a convenience when a lot of the processing is taking place in real time. The input matrix structure that was decided upon during Holzl's investigation, [(Directions x Subjects)(Frequencies x 2)], was modified slightly to work for a single user to become: [Directions x (Frequencies x 2)]. In practical terms, an entry from the CIPIC HRIR database would be transformed into the frequency domain, resulting in an HRTF dataset in the form [Left/Right (2) x Azimuths (25) x Elevations (50) x Frequencies (101)]. This is then restructured into the above form, resulting in a structure [(Azimuths, Elevations (1250)) x (Frequencies, Left/Right (202))] in size. This structure is intuitive in terms of how the original values map to the new one, a quality that carries over even after the structure has been transformed with PCA. Performing PCA on this matrix singles out the frequency bins that contain the most variance over all source directions. The resulting [PCW x PC] matrix maps to [Directions x PCs] where the PCs are the frequency bins of greatest variance and the directions are the PCWs. The benefit of this resulting structure is that it becomes very easy to modify the PCWs that relate to the position of a sound source. So for my implementation this means that it is simple to map the degree to which a user is able to locate a sound source to a relevant PCW within each PC. Because of this ability to match sound sources with principal component weights, coupled with the fact that PCW modifications were to be automated rather than performed manually, I chose not to model the resulting PCWs using spherical harmonics, further simplifying the individualisation process. 

The PCA model produced by this input matrix also allows for around ninety percent of the variance in the to be described by 10 PCs, greatly limiting the number of variables that can be modified. This means that to adjust the perceived source of a sample the algorithm needs to only modify ten PCWs, one for each PCW that corresponds to that source position in each PC. 

Core to simulated annealing (SA) search is that the degree of randomness, or the range of potential child states, is reduced as the current state gets closer to the goal state. In this implementation of the algorithm the value that is used to update the PCW is derived from user's localisation error, so the closer the user is to locating the sound source correctly, the smaller the change that is made to the PCW. It is worth noting that this search method is essentially running many individual searches - either in sequence or in parallel - in order to find the optimal value for every individual source position. 

Modifying PCWs like this does mean that generating an individualised HRTF set would take at least 1250 seperate measurements, something that is almost as laborious as the standard measurement procedure. So to expedite the process, or at least to affect a greater amount of the HRTF with each modification, the PCWs for the eight source positions directly around the source being tested will also be modified by the same value, halved. This may mean that subsequent modifications partially overwrite previous ones, but it seems preferable to expedite the process at least while data on the relationship between PCWs and localisation errors is so limited. 

Because there will be so much time betweeen each subsequent alteration made to a given PCW, the details of each alteration made are stored for future measurements from the same source position. This means that each time the algorithm begins to modify a set of PCWs, it first checks the existing database for previous adjustments. If extant, the previous error value, the modifier value for each PCW, and the change direction (whether the modifier was added to or subtracted from the weight) are all returned, and can inform this change. The direction of change is informed by whether or not this localisation attempt fared better than the previous one. If it did, then the change is made in the same direction. If it did not, the change is made back the other way. In the event that a change is reversed, this system will ensure that the next change is made in the same direction. 

[CONFIRM HOW THE MEMORY ASPECT WORKS, AM I MODIFYING ALL PCWS AT ONCE IN THE SAME DIRECTION BY THE SAME VALUE? OR STORING AN ARRAY OF MODIFICATION VALUES? BE CLEAR]

\subsubsection{Implementation}
The project that has been produced to test this proposed method is formed of two three parts: A front-end VR test environment that manages the positions of the sound sources and processes the audio samples, and passes data to another module for processing. This individualisation module forms the core of the implementation, as it handles the deconstruction, modification, and reconstruction of the HRTF data. This module is then tied to an in-memory key-value datastore that holds the starting HRTFs, intermediate results, and an archive of the changes made so far. 

\paragraph{Technologies}
The bulk of the implementation is written in Python\citep{python lol}, making liberal use of the SciPy\citep{scipy and numpy} set of libraries to handle the majority of processing involved. Also in use is the simplejson\citep{simplejson} module, used to encode both HRTF data and logs extensively. To handle principal components analysis, the scikit-learn\citep{scikit learn} PCA class from the decomposition module was employed. Lastly, interfacing with the database was done using the lmdb module\citep{lmdb}. Symas' Lightning Memory-mapped Database \citep{lmdb db} was chosen both because it is a simple key-value datastore that doesn't require the kind of strictly defined strucure a relational database might, and because the entire contents of it can be loaded into memory when the program is started, making fetch and store operations comparatively rapid - useful when working in real-time. Finally, the test environment was produced sing the Unity game engine \citep{•}, with the GoogleVR SDK \citep{•} and the Final Wireframe \citep{•} shader pack.

\paragraph{Process}
Typically the individualisation process would run as follows, beginning with the user or participant in the VR space that is being used to generate the data. 

\begin{itemize}
\item The user is played an audio cue, from a source position that is generated at random but that does correspond to a source position available in the database - in the case of CIPIC, this is anything from -45 degrees and up.
\item 
\end{itemize}
