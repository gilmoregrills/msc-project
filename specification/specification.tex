% File: specification.tex
% Created: Tuesday 24th January
% 
% MSc Project Specification

\documentclass[10pt, oneside, a4paper, draft]{scrartcl}

\begin{document}
\bibliography{/home/gilmoregrills/Documents/library.bib}
\title{HRTF Individualisation Based on Localization Errors in 3D Space}
\subtitle{Specification}
\author{Robin Yonge}
\date{Febuary 2017}
\maketitle

\renewcommand{\abtractname}{Introduction} %maybs change to blank
\begin {abstract}
Virtual reality\ignore{Computer-mediated reality \cite{S.Mann199}} has come to represent the dominant prediction for the near future of human-computer interaction. If this prediction is to come to pass, then it is necessary that sufficient consideration is given toward the fidelity of both the visual and the auditory aspects of VR and AR technology. Currently, most implementations of spatial audio in VR and AR applications are based around the use of HRTFs. Mostof these implementations use one of a few publically available datasets, derived from measurements performed on either human subjects or a model such a the KEMAR\cite{Algazi2001}\cite{Gardner1994}. For the use of VR/AR to become commonplace the experience must be universal. Because spatial audio relies on complex measurements of real-world subjects (be they flesh and blood or not) applications tend to use average or neutral HRTFs, ones that should work for the largest number of people. Of course, by dint of being average there are many people for whom these do not work. In order to achieve a universal experience, then, it is necessary to find a process for  simple individualization of spatial audio that is sufficiently simple, and can be performed by the end user as a part of the standardsetup of any VR/AR device or application.
\end {abstract}

\section*{p1}
The goal of this project is to produce a working prototype of an application that takes as input a neutral set of HRTF measurements, and returns an individualised set. For a process like this to achieve widespread use it must be simple enough that it can be performed by any given user with nothing more than a head-mounted display, and intuitive enough that it requires no in-depth knowledge of spatial audio. The output HRTF set should also provide a noticeable improvement in the ability of the user to locate sound sources in a virtual 3D environment.

Previous efforts to produce individualized HRTFs through either modification of an existing neutral set or total synthesis have focused primarily on a few different techniques. The most common method is based on a structural HRTF model \cite{PhillipBrown1998, Raykar2003}, in which different characteristics of the HRTF are linked to physical characteristics of the pinna, torso, and head. Implementations based on this model have been varied \cite{Tashev2014, Xu2007, DmitryN.ZotkinJaneHwangRamaniDuraiswami2003}, but the model's reliance on even limited sets of precise anthropometric measurements makes it impractical for my purposes - despite having the potential to be much more efficient than traditional HRIR measurement. 

>>include other methods???

More recent studies have involved applying Principal Component Analysis (PCA) to HRTF data \cite{Holzl2012a}, in order to identify the spectral features that have the greatest effect upon the incoming sound signal. Once identified, these Principal Components (PCs) can potentially be adjusted in order to produce a more individualised HRTF \cite{Fink2012,Holzl2014a}. Typically upwards of ninety percent of the information in a given HRTF is based on between 5 and 10 PCs, so through the adjustment of Principal Component Weights (PCWs) it should be possible to individualise HRTFs with a reasonably high degree of accuracy. The combined efficacy and efficiency of this method makes it well-suited to my proposed implementation, and as such this will be the model I will be using to perform the individualisation. 

Once they have been identified, I will need data from the user on which to base any adjustment of PCWs. Because on my stated goal of producing an intuitive simple method, I will be performing the individualisation based on the user's own errors in localisation. This method will require me to identify, where possible, points of correlation between PCs and localisation, and as such this will be one of my primary research foci going forward. In order to gather this data from the user it is most practical to derive it from their own localisation errors in a virtual environment. Ideally, it will involve placing the user in virtual space, and asking them to identify the source of the audio cues that will be played to them. This helps to create an intuitive user experience, and giving the user a reticle to point at the perceived source gives precise data upon which to make adjustments. As an additional stretch goal of sorts, building a virtual environment that makes the calibration process more engaging than simply dropping the user into an empty void and blasting them with sound cues. 

Based on these decisions, the finished application will be comprised of three parts: The database/files containing the source HRTF data and individualised copies, a frontend interface to be viewed through a head-mounted display, and a module that takes localisation data from the frontend and makes adjustments based on it to the HRTF data. 

notes: 

Might still want to mention the incorporation of search algorithms/ML of a sort in deciding what PCWs to adjust

\section*{p2}

what technologies will I use and why?

The bulk of the implementation will be written in Python, particularly the core HRTF-adjusting module. Speed isn't too much of an issue the amount of processing to be done at one time will likely not exceed the capabilities of any device sufficiently powerful to run an HMD, and the access to the SciPy libraries\cite{scipy website???} will make reading, visualising, and editing the files containing the HRTF data much simpler. The other option was to use MATLAB, which would require me to learn a whole new language for very little benefit - it would allow me to skip one step in in processing the files, as most of the publicly-available HRTF datasets are stored in .mat files.



Building the core hrtf-modifying module in Python and SciPy, that should be all that needs. Would like this module to have a nice sort-of API, to demonstrate the ease with which it could be incorporated into another product and to make it easier for me to build stuff on top of it.   

The database used looks like it will likely be the CIPIC KEMAR set, because it's more well-documented than the other newer databases, and of a higher resolution? Also looks like the AIR and IRCAM ones don't have a mannequin head version and using a human dataset seems like a bad starting point?  

The UI/frontend I'm still not 100percent on, I know Unity and it would be a weekend's build time, but I'm not sure I can necessarily get the data I need from it out to my python module and I don't know if building it as a Unity plugin is an option? Either that or I learn me some simple OpenGL. 

When it comes to processing the audio in the scene itself I think I'll just use something simple like fmod because simpler with a spatial audio/convolution plugin because I don't *think* I want to fuck with real DSP. ITD/ILD/filtering is all contained in the hrtf so the audio approach doesn't need to be complex. 

For rudimentary testing I'll record a few sample angles/elevations with the binaural microphones on myself and then test them against the individualised hrtfs for the same directions. After that I can test it purely based on the error rate of the users in the final product. 

\section*{}
Work plan, timeframes, etc

Minimum Viable Product: 
Python hrtf-editing module, with a blank void VR frontend that's full of sound sources. 

Ideal:
Python hrtf-editing module with visualisations, nice fancy VR frontend that turns it into a game of locate-the-object.



\end{document}


